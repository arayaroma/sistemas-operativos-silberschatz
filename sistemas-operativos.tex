\documentclass{article}
\usepackage{amsmath}

\title{Apuntes de Sistemas Operativos}
\date{2023-05-30}
\author{Daniel Araya Rom\'{a}n}

\begin{document}
\maketitle
\newpage

\section*{1. Introducci\'{o}n}
\paragraph*{}
\normalsize

Un \textbf{sistema operativo} es un programa que administra el hardware de una computadora.
Act\'{u}a como intermediario entre el usuario y el hardware. Un aspecto sorprendente de los 
sistemas operativos es la gran variedad de formas en que llevan a cabo estas tareas. Los 
sistemas operativos \textit{mainframe} est\'{a}n dise\~{n}ados para optimizar el uso del hardware.
Algunos est\'{a}n dise\~{n}ados para ser pr\'{a}cticos, otros para ser eficientes y otros para ser
ambas cosas. Antes de adentrarnos en los detalles de los sistemas operativos, es importante
entender acerca de la estructura del sistema. Dado que un sistema operativo es un software grande
y complejo, debe crearse pieza por pieza. En este cap\'{i}tulo se describe los principales componentes
de un sistema operativo.

\subsection*{1.1 Qu\'{e} hace un sistema operativo?}
Un sistema operativo es un sistema inform\'{a}tico que puede dividirse en cuatro componentes:
\textit{hardware, sistema operativo, programas de aplicaci\'{o}n y usuarios}.

El \textbf{hardware}, la \textbf{unidad central de procesamiento} (CPU), la \textbf{memoria} y los
\textbf{dispositivos de entrada/salida} (E/S), proporcionan los recursos b\'{a}sicos de c\'{o}mputo
al sistema. Los \textbf{programas de aplicaci\'{o}n}, como los procesadores de texto, las hojas de
c\'{a}lculo, los compiladores y los navegadores web, definen las formas en que estos recursos se
emplean para resolver los problemas inform\'{a}ticos de los usuarios. 

\begin{center}
    \textbf{Analog\'{i}a} \\
    Sistema operativo $\rightarrow$ Gobierno $\rightarrow$ Entorno de programas $\rightarrow$ 
    Trabajo \'{u}til 
\end{center}

\subsubsection*{1.1.1 Punto de vista del usuario}
La mayor\'{i}a de usuarios disponen de un monitor, teclado, un rat\'{o}n, una unidad de sistema.
Un sistema as\'{i} se dise\~{n}a para que el usuario \textbf{monopolice} sus recursos. El objetivo
es maximizar el trabajo que el usuario realice. En este caso tiene que dise\~{n}arse para que sea
de f\'{a}cil uso.

En otros casos, un usuario se sienta en frente a un terminal conectado a un \textbf{mainframe} o una
\textbf{microcomputadora}. Otros usuarios acceden simult\'{a}neamente a trav\'{e}s de otros terminales.
Estos usuarios comparten recursos y pueden intercambiar informaci\'{o}n. En tal caso, el sistema operativo
se dise\~{n}a para maximizar la utilizaci\'{o}n de recursos, de modo que cada usuario disponga s\'{o}lo de 
una parte equitativa que le corresponde.

En otros casos, los usuarios usan \textbf{estaciones de trabajo} conectadas a redes de otras estaciones de
trabajo y servidores. Los usuarios tienen recursos dedicados, pero tambi\'{e}n tienen recursos compartidos
como la red y los servidores. Por tanto su sistema operativo est\'{a} dise\~{n}ado para llegar a un compromiso
entre la usabilidad individual y la utilizaci\'{o}n de recursos.

\subsubsection*{1.1.2 Vista del sistema}
El sistema operativo es el programa m\'{a}s \'{i}ntimamente relacionado con el hardware. Podemos ver al sistema
operativo como un \textbf{asignador de recursos}. El sistema operativo \textbf{act\'{u}a} como el administrador
de estos recursos. Al enfrentarse a numerosas y posibles conflictivas solicitudes de recursos, el sistema operativo
debe de decidir c\'{o}mo asignarlos a programas y usuarios espec\'{i}ficos, de modo que el computador opere de manera
eficiente y equitativa.

Un punto de vista que difiere al sistema operativo, hace hincapi\'{e} en la necesidad de controlar dispositivos de 
E/S y programas de usuario. Un sistema operativo es un \textbf{programa de control}.

\subsubsection*{1.1.3 Definici\'{o}n de sistemas operativos}
No hay una definici\'{o}n de sistema operativo que sea completamente adecuada. Estos existen porque ofrecen una forma
razonable de resolver el problema de crear un sistema inform\'{a}tico utilizable. El \textbf{objetivo principal} de
las computadoras es \textbf{ejecutar} programas de usuario y resolver problemas del mismo f\'{a}cilmente. Con este
objetivo se construye el hardware de la computadora. Ya que el hardware no es f\'{a}cil de usar, se desarrollaron 
programas de aplicaci\'{o}n. Estos programas requieren operacaiones comunes, y estas se incorporan en una pieza de software
que es el sistema operativo.

Adem\'{a}s, no hay ninguna definici\'{o}n universalmente aceptada sobre qu\'{e} forma parte de un sistema operativo. Las
caracter\'{i}sticas var\'{i}an de un sistema a otro. Algunos sistemas operativos ocupan 1 megabyte de espacio y
no proporcionan ni un editor a pantalla completa, mientras que otros necesitan gigabytes de espacio y est\'{a}n
completamente basados en sistemas gr\'{a}ficos de ventanas.
\medbreak

\begin{center}
    \textbf{Unidades de medidad} 
\end{center}
\begin{align*}
    1 \text{ bit} &= 0 \text{ or } 1 \\ 
    1 \text{ byte} &= 8 \text{ bits} \\ 
    1 \text{ kilobyte} &= 1024^1 \text{ bytes} \\ 
    1 \text{ megabyte} &= 1024^2 \text{ bytes} \\ 
    1 \text{ gigabyte} &= 1024^3 \text{ bytes}
\end{align*}

Otra definici\'{o}n com\'{u}n es que un sistema operativo es aquel programa que se ejecuta continuamente en la
computadora (usualemente denominado \textbf{\textit{kernel}}), siendo todo lo dem\'{a}s programas del sistema y
programas de aplicaci\'{o}n.

\subsection*{1.2 Organizaci\'{o}n de una computadora}
Antes de entender como funciona una computadora, debemos entender su \textbf{estructura}.

\subsubsection*{1.2.1 Funcionamiento de una computadora}
Una computadora moderna de prop\'{o}sito general consta de una o m\'{a}s CPU y una serie de controladoras de dispositivo
conetadas a trav\'{e}s de un \textbf{bus com\'{u}n} que proporciona acceso a la \textbf{memoria compartida}. Cada controladora
de dispositivo se encarga de un tipo espec\'{i}fico de dispositivo, por ejemplo, unidades de disco, dispositivos de audio y  pantallas
de video. 

La CPU y estas controladoras pueden funcionar de forma concurrente, compitiendo por los ciclos de memoria. Para asegurar
el acceso de forma ordenada a la memoria compartida, se proporciona una controladora de memoria cuya funci\'{o}n es \textbf{sincronizar}
el acceso a la misma.

Para que una computadora empiece a funcionar, es necesario que tenga un programa de inicio de ejecutar. Este \textbf{programa de arranque}
suele ser simple. Normalmente se almacena en la memoria \textbf{ROM (read-only memory)} o en una memoria 
\textbf{EEPROM (electrically erasable programmable read-only memory)}; conocida con el t\'{e}rmino general de \textbf{firmware}.
El programa de arranque debe saber c\'{o}mo cargar el sistema operativo e iniciar la ejecuci\'{o}n del mismo. Para esto debe localizar
y cargar en memoria el kernel (n\'{u}cleo) del sistema operativo. Despu\'{e}s, el sistema operativo comienza ejecutando el 
\textbf{primer proceso}, como por ejemplo \textit{init} y espera a que se produzca alg\'{u}n suceso.

La ocurrencia de un suceso normalmente se indica mediante una \textbf{interrupci\'{o}n}, bien del hardware o software. El hardware
puede activar una interrupci\'{o}n en cualquier instante enviando una se\~{n}al a la CPU, normalmente a trav\'{e}s del bus del
sistema. El software puede activar una interrupci\'{o}n mediante una operaci\'{o}n especial llamada de \textbf{llamada al sistema}.



\end{document}