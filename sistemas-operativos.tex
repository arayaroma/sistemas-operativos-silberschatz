\documentclass{article}
\usepackage{amsmath}
\usepackage[a4paper, hmargin=1in,vmargin=1in]{geometry}
\usepackage[edges]{forest}

\title{Apuntes de Sistemas Operativos}
\date{2023-05-30}
\author{Daniel Araya Rom\'{a}n}

\begin{document}
\maketitle
\newpage

\section*{1. Introducci\'{o}n}
\paragraph*{}
\normalsize

Un \textbf{sistema operativo} es un programa que administra el hardware de una computadora.
Act\'{u}a como intermediario entre el usuario y el hardware. Un aspecto sorprendente de los 
sistemas operativos es la gran variedad de formas en que llevan a cabo estas tareas. Los 
sistemas operativos \textit{mainframe} est\'{a}n dise\~{n}ados para optimizar el uso del hardware.
Algunos est\'{a}n dise\~{n}ados para ser pr\'{a}cticos, otros para ser eficientes y otros para ser
ambas cosas. Antes de adentrarnos en los detalles de los sistemas operativos, es importante
entender acerca de la estructura del sistema. Dado que un sistema operativo es un software grande
y complejo, debe crearse pieza por pieza. En este cap\'{i}tulo se describe los principales componentes
de un sistema operativo.

\subsection*{1.1 Qu\'{e} hace un sistema operativo?}
Un sistema operativo es un sistema inform\'{a}tico que puede dividirse en cuatro componentes:
\textit{hardware, sistema operativo, programas de aplicaci\'{o}n y usuarios}.
El \textbf{hardware}, la \textbf{unidad central de procesamiento} (CPU), la \textbf{memoria} y los
\textbf{dispositivos de entrada/salida} (E/S), proporcionan los recursos b\'{a}sicos de c\'{o}mputo
al sistema. Los \textbf{programas de aplicaci\'{o}n}, como los procesadores de texto, las hojas de
c\'{a}lculo, los compiladores y los navegadores web, definen las formas en que estos recursos se
emplean para resolver los problemas inform\'{a}ticos de los usuarios. 

\begin{center}
    \textbf{Analog\'{i}a} \\
    Sistema operativo $\rightarrow$ Gobierno $\rightarrow$ Entorno de programas $\rightarrow$ 
    Trabajo \'{u}til 
\end{center}

\subsubsection*{1.1.1 Punto de vista del usuario}
La mayor\'{i}a de usuarios disponen de un monitor, teclado, un rat\'{o}n, una unidad de sistema.
Un sistema as\'{i} se dise\~{n}a para que el usuario \textbf{monopolice} sus recursos. El objetivo
es maximizar el trabajo que el usuario realice. En este caso tiene que dise\~{n}arse para que sea
de f\'{a}cil uso.

En otros casos, un usuario se sienta en frente a un terminal conectado a un \textbf{mainframe} o una
\textbf{microcomputadora}. Otros usuarios acceden simult\'{a}neamente a trav\'{e}s de otros terminales.
Estos usuarios comparten recursos y pueden intercambiar informaci\'{o}n. En tal caso, el sistema operativo
se dise\~{n}a para maximizar la utilizaci\'{o}n de recursos, de modo que cada usuario disponga s\'{o}lo de 
una parte equitativa que le corresponde.

En otros casos, los usuarios usan \textbf{estaciones de trabajo} conectadas a redes de otras estaciones de
trabajo y servidores. Los usuarios tienen recursos dedicados, pero tambi\'{e}n tienen recursos compartidos
como la red y los servidores. Por tanto su sistema operativo est\'{a} dise\~{n}ado para llegar a un compromiso
entre la usabilidad individual y la utilizaci\'{o}n de recursos.

\subsubsection*{1.1.2 Vista del sistema}
El sistema operativo es el programa m\'{a}s \'{i}ntimamente relacionado con el hardware. Podemos ver al sistema
operativo como un \textbf{asignador de recursos}. El sistema operativo \textbf{act\'{u}a} como el administrador
de estos recursos. Al enfrentarse a numerosas y posibles conflictivas solicitudes de recursos, el sistema operativo
debe de decidir c\'{o}mo asignarlos a programas y usuarios espec\'{i}ficos, de modo que el computador opere de manera
eficiente y equitativa.
Un punto de vista que difiere al sistema operativo, hace hincapi\'{e} en la necesidad de controlar dispositivos de 
E/S y programas de usuario. Un sistema operativo es un \textbf{programa de control}.

\subsubsection*{1.1.3 Definici\'{o}n de sistemas operativos}
No hay una definici\'{o}n de sistema operativo que sea completamente adecuada. Estos existen porque ofrecen una forma
razonable de resolver el problema de crear un sistema inform\'{a}tico utilizable. El \textbf{objetivo principal} de
las computadoras es \textbf{ejecutar} programas de usuario y resolver problemas del mismo f\'{a}cilmente. Con este
objetivo se construye el hardware de la computadora. Ya que el hardware no es f\'{a}cil de usar, se desarrollaron 
programas de aplicaci\'{o}n. Estos programas requieren operacaiones comunes, y estas se incorporan en una pieza de software
que es el sistema operativo.
Adem\'{a}s, no hay ninguna definici\'{o}n universalmente aceptada sobre qu\'{e} forma parte de un sistema operativo. Las
caracter\'{i}sticas var\'{i}an de un sistema a otro. Algunos sistemas operativos ocupan 1 megabyte de espacio y
no proporcionan ni un editor a pantalla completa, mientras que otros necesitan gigabytes de espacio y est\'{a}n
completamente basados en sistemas gr\'{a}ficos de ventanas.
\medbreak

\begin{center}
    \textbf{Unidades de medidad} 
\end{center}
\begin{align*}
    1 \text{ bit} &= 0 \text{ or } 1 \\ 
    1 \text{ byte} &= 8 \text{ bits} \\ 
    1 \text{ kilobyte} &= 1024^1 \text{ bytes} \\ 
    1 \text{ megabyte} &= 1024^2 \text{ bytes} \\ 
    1 \text{ gigabyte} &= 1024^3 \text{ bytes}
\end{align*}

Otra definici\'{o}n com\'{u}n es que un sistema operativo es aquel programa que se ejecuta continuamente en la
computadora (usualemente denominado \textbf{\textit{kernel}}), siendo todo lo dem\'{a}s programas del sistema y
programas de aplicaci\'{o}n.

\subsection*{1.2 Organizaci\'{o}n de una computadora}
Antes de entender como funciona una computadora, debemos entender su \textbf{estructura}.

\subsubsection*{1.2.1 Funcionamiento de una computadora}
Una computadora moderna de prop\'{o}sito general consta de una o m\'{a}s CPU y una serie de controladoras de dispositivo
conetadas a trav\'{e}s de un \textbf{bus com\'{u}n} que proporciona acceso a la \textbf{memoria compartida}. Cada controladora
de dispositivo se encarga de un tipo espec\'{i}fico de dispositivo, por ejemplo, unidades de disco, dispositivos de audio y  pantallas
de video. 
La CPU y estas controladoras pueden funcionar de forma concurrente, compitiendo por los ciclos de memoria. Para asegurar
el acceso de forma ordenada a la memoria compartida, se proporciona una controladora de memoria cuya funci\'{o}n es \textbf{sincronizar}
el acceso a la misma.

Para que una computadora empiece a funcionar, es necesario que tenga un programa de inicio de ejecutar. Este \textbf{programa de arranque}
suele ser simple. Normalmente se almacena en la memoria \textbf{ROM (read-only memory)} o en una memoria 
\textbf{EEPROM (electrically erasable programmable read-only memory)}; conocida con el t\'{e}rmino general de \textbf{firmware}.
El programa de arranque debe saber c\'{o}mo cargar el sistema operativo e iniciar la ejecuci\'{o}n del mismo. Para esto debe localizar
y cargar en memoria el kernel (n\'{u}cleo) del sistema operativo. Despu\'{e}s, el sistema operativo comienza ejecutando el 
\textbf{primer proceso}, como por ejemplo \textit{init} y espera a que se produzca alg\'{u}n suceso.

La ocurrencia de un suceso normalmente se indica mediante una \textbf{interrupci\'{o}n}, bien del hardware o software. El hardware
puede activar una interrupci\'{o}n en cualquier instante enviando una se\~{n}al a la CPU, normalmente a trav\'{e}s del bus del
sistema. El software puede activar una interrupci\'{o}n mediante una operaci\'{o}n especial llamada de \textbf{llamada al sistema}.
Cuando la CPU se interrumpe, deja lo que est\'{a} haciendo e inmediatamente transfiere la
ejecuci\'{o}n a una posici\'{o}n fijada. Normalmente contiene la direcci\'{o}n de inicio
donde se encuentra la rutina de servicio a la interrupci\'{o}n. Luego de ejecutar esta rutina
la CPU reanuda la operaci\'{o}n que estuviera haciendo.

El m\'{e}todo m\'{a}s simple para tratar la transferencia consiste en invocar una rutina
gen\'{e}rica para examinar la informaci\'{o}n de la interrupci\'{o}n. Sin embargo estas
interrupciones deben de tratarse r\'{a}pidamente, y este procedimiento es algo lento. 
Solamente solo es posible un n\'{u}mero predefinido de interrupciones; o usar otro sistema
consistente en disponer una tabla de punteros a las rutinas de interrupci\'{o}n. Proporcionando
la velocidad necesaria, de forma indirecta se llama a trav\'{e}s de la tabla. Sin necesidad
de una rutina intermedia, generalmente la tabla de punteros se almacena en una posici\'{o}n
inferior de la memoria. Este \textbf{vector de interrupciones}, se indexa mediante un n\'{u}mero 
un\'{i}co que se proporciona con la solicitud de interrupci\'{o}n, para obtener la
direcci\'{o}n de la rutina de servicio a la interrupci\'{o}n para el dispositivo correspondiente.
Sistemas operativos tan diferentes como \textbf{Windows y UNIX} manejan las interrupciones
de esta forma.

Esta arquitectura debe de almacenar la direcci\'{o}n de la instrucci\'{o}n interrumpida.
Las arquitectuas m\'{a}s recientes almacenan la direcci\'{o}n de retorno en la \textbf{pila del
sistema}. Si la rutina e interrupci\'{o}n necesita modificar el estado del procesador,
debe guardar expl\'{i}citamente el estado actual y luego restaurar dicho estado antes
de volver. Despu\'{e}s la direcci\'{o}n de retorno guardada se carga en el contador de
programa y la ejecuci\'{o}n contin\'{u}a como si nada hubiera pasado.

\subsection*{1.2.2 Estructura de almacenamiento}
Los programas de una computadora deben hallarse en la memoria principal (memoria \textbf{RAM}),
\textit{random-access memory}, para ser ejecutados. El procesador puede acceder directamente.
Habitualmente, se implementa con una tecnolog\'{i}a de semiconductores denominada \textbf{DRAM 
(dynamic random-access memory)}, que forma una matriz de palabras de memoria. Cada palabra tiene
su propia direcci\'{o}n, se interact\'{u}a a trav\'{e}s de una secuencia de carga (load) o 
almacenamiento (store) de instrucciones en direcciones espec\'{i}ficas de memoria. La instrucci\'{o}n
load mueve una palabra desde la memoria principal a un registro interno de la CPU, mientras
que la instrucci\'{o}n store mueve una palabra desde el registro interno a la memoria principal.

Un ciclo t\'{i}pico \textbf{instrucci\'{o}n-ejecuci\'{o}n}, cuando se ejecuta en un sistema de
la arquitectura de \textbf{Von Neumann}, primero se extrae una instrucci\'{o}n de memoria y
se almacena dicha instrucci\'{o}n en el \textbf{registro de instrucciones}. Continuamente la
instrucci\'{o}n se decodifica y puede dar lugar a que se extraigan los operandos de la memoria
y se almacenen en alg\'{u}n registro interno. Despu\'{e}s de ejecutar la instrucci\'{o}n con
los operandos necesarios, el resultado se almacena de nuevo en memoria.

Naturalmente la unidad de memoria solo ve un flujo de direcciones de memoria; no sabe c\'{o}mo
se an generado o qu\'{e} son. Se puede ignorar c\'{o}mo genera un programa una direcci\'{o}n de
memoria. Solo interesa la secuencia de direcciones de memoria generadas por el programa en ejecuci\'{o}n.
Idealmente, es deseable que los programas y los datos residan en la memoria principal de
forma permanente. Normalmente no es posible por dos razones:

\begin{enumerate}
    \item Normalmente, la memoria principa es demasiado peque\~{n}a para contener todos los
    programas y datos necesarios de forma permanente.
    \item La memoria principal es vol\'{a}til; pierde su contenido cuando se le quita la alimentaci\'{o}n.
\end{enumerate}

Por tanto la mayor parte de los sistemas inform\'{a}ticos requieren de almacenamiento secundario,
como una extensi\'{o}n de la memoria principal. Este puede almacenar grandes cantidades de datos
de manera permanente.
El dispositivo de almacenamiento secundario m\'{a}s com\'{u}n es el \textbf{disco magn\'{e}tico},
proporcionando un sistema de almacenamiento tanto para programas como para datos. La mayor\'{i}a
de los programas se almacenan en un disco hasta que se cargan en memoria. Por lo que una apropiada
administraci\'{o}n del almacenamiento en disco es de importancia crucial en un sistema inform\'{a}tico
como se puede ver en el \textit{cap\'{i}tulo 12}.

No obstante, en un sentido amplio, la estructura de almacenamiento consta de registros, memoria principal,
y discos magn\'{e}ticos, solo es uno de los muchos sistemas de almacenamiento. Otros sistemas pueden
incluir memoria cach\'{e}, CD-ROM, cintas magn\'{e}ticas, etc. Cada sistema proporciona las funciones
b\'{a}sicas para guardar datos y mantenerlos hasta que estos sean recuperados en un instante posterior.
Las principales diferencias entre los sistemas de almacenamiento son la \textbf{velocidad}, el
\textbf{coste}, el \textbf{tama\~{n}o} y la \textbf{volatilidad}.

\subsection*{1.2.3 Estructura de E/S}


\end{document}